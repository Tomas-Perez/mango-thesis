\chapter*{Sommario}
\addcontentsline{toc}{chapter}{Sommario}
\lettrine{I}{l} crescente numero di applicazioni ad alte prestazioni, dalle simulazioni fisiche alla finanza computazionale, fino alle applicazioni di intelligenza artificiale ha portato ad un incremento della richiesta di sistemi per il calcolo ad alte prestazioni (HPC). Il progetto MANGO è stato creato per permettere lo sviluppo di applicazioni per sistemi HPC eterogenei. Fino ad ora, MANGO ha offerto supporto per una architettura hardware specializzata, impiegata per l'esplorazione di soluzioni many-core. Questo lavoro di tesi fornisce varie estensioni alla piattaforma MANGO: una API Python, il supporto per la compilazione Just-In-Time (JIT) dei kernel computazionali, e infine il supporto per l'accelerazione su GPU, ovvero il sistema di calcolo parallelo eterogeneo più comune. L'introduzione del supporto per le GPU ha portato ad una profonda ristrutturazione dell'architettura software, con l'introduzione di uno strato di astrazione dell'hardware (HAL) e una riformulazione del modulo in interfaccia.
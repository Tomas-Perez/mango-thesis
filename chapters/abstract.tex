\chapter*{Abstract}
\addcontentsline{toc}{chapter}{Abstract}
\lettrine{T}{he} increasing number of processing-power hungry applications, from traditional physics and financial computing solutions, to the now raising wave of machine and deep learning, has led to a surge in demand for High Performance Computing (HPC) systems.
The MANGO project was introduced with the objective of enabling the development of user applications on heterogeneous HPC systems.
Up to this point, MANGO offered support for a custom hardware architecture, used to explore manycore architectures for HPC systems.
This thesis focuses on a number of additions to the MANGO platform, starting with a Python language API, followed by the Just In Time compilation of computing kernels, and culminating in the introduction of GPU accelerators: highly parallel devices that are commonly found in heterogeneous systems. 
The incorporation of GPU support resulted in a restructure of the MANGO software stack, with the addition of a Hardware Abstraction Layer and a complete rework of the user-facing module.


